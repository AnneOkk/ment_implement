% Options for packages loaded elsewhere
\PassOptionsToPackage{unicode}{hyperref}
\PassOptionsToPackage{hyphens}{url}
%
\documentclass[
  man]{apa7}
\usepackage{amsmath,amssymb}
\usepackage{lmodern}
\usepackage{iftex}
\ifPDFTeX
  \usepackage[T1]{fontenc}
  \usepackage[utf8]{inputenc}
  \usepackage{textcomp} % provide euro and other symbols
\else % if luatex or xetex
  \usepackage{unicode-math}
  \defaultfontfeatures{Scale=MatchLowercase}
  \defaultfontfeatures[\rmfamily]{Ligatures=TeX,Scale=1}
\fi
% Use upquote if available, for straight quotes in verbatim environments
\IfFileExists{upquote.sty}{\usepackage{upquote}}{}
\IfFileExists{microtype.sty}{% use microtype if available
  \usepackage[]{microtype}
  \UseMicrotypeSet[protrusion]{basicmath} % disable protrusion for tt fonts
}{}
\makeatletter
\@ifundefined{KOMAClassName}{% if non-KOMA class
  \IfFileExists{parskip.sty}{%
    \usepackage{parskip}
  }{% else
    \setlength{\parindent}{0pt}
    \setlength{\parskip}{6pt plus 2pt minus 1pt}}
}{% if KOMA class
  \KOMAoptions{parskip=half}}
\makeatother
\usepackage{xcolor}
\IfFileExists{xurl.sty}{\usepackage{xurl}}{} % add URL line breaks if available
\IfFileExists{bookmark.sty}{\usepackage{bookmark}}{\usepackage{hyperref}}
\hypersetup{
  pdftitle={Implementation Issues AI - Mental Healthcare},
  pdfauthor={Anne-Kathrin Kleine1 \& },
  pdflang={en-EN},
  pdfkeywords={AI, Mental Health},
  hidelinks,
  pdfcreator={LaTeX via pandoc}}
\urlstyle{same} % disable monospaced font for URLs
\usepackage{graphicx}
\makeatletter
\def\maxwidth{\ifdim\Gin@nat@width>\linewidth\linewidth\else\Gin@nat@width\fi}
\def\maxheight{\ifdim\Gin@nat@height>\textheight\textheight\else\Gin@nat@height\fi}
\makeatother
% Scale images if necessary, so that they will not overflow the page
% margins by default, and it is still possible to overwrite the defaults
% using explicit options in \includegraphics[width, height, ...]{}
\setkeys{Gin}{width=\maxwidth,height=\maxheight,keepaspectratio}
% Set default figure placement to htbp
\makeatletter
\def\fps@figure{htbp}
\makeatother
\setlength{\emergencystretch}{3em} % prevent overfull lines
\providecommand{\tightlist}{%
  \setlength{\itemsep}{0pt}\setlength{\parskip}{0pt}}
\setcounter{secnumdepth}{-\maxdimen} % remove section numbering
% Make \paragraph and \subparagraph free-standing
\ifx\paragraph\undefined\else
  \let\oldparagraph\paragraph
  \renewcommand{\paragraph}[1]{\oldparagraph{#1}\mbox{}}
\fi
\ifx\subparagraph\undefined\else
  \let\oldsubparagraph\subparagraph
  \renewcommand{\subparagraph}[1]{\oldsubparagraph{#1}\mbox{}}
\fi
\newlength{\cslhangindent}
\setlength{\cslhangindent}{1.5em}
\newlength{\csllabelwidth}
\setlength{\csllabelwidth}{3em}
\newlength{\cslentryspacingunit} % times entry-spacing
\setlength{\cslentryspacingunit}{\parskip}
\newenvironment{CSLReferences}[2] % #1 hanging-ident, #2 entry spacing
 {% don't indent paragraphs
  \setlength{\parindent}{0pt}
  % turn on hanging indent if param 1 is 1
  \ifodd #1
  \let\oldpar\par
  \def\par{\hangindent=\cslhangindent\oldpar}
  \fi
  % set entry spacing
  \setlength{\parskip}{#2\cslentryspacingunit}
 }%
 {}
\usepackage{calc}
\newcommand{\CSLBlock}[1]{#1\hfill\break}
\newcommand{\CSLLeftMargin}[1]{\parbox[t]{\csllabelwidth}{#1}}
\newcommand{\CSLRightInline}[1]{\parbox[t]{\linewidth - \csllabelwidth}{#1}\break}
\newcommand{\CSLIndent}[1]{\hspace{\cslhangindent}#1}
\ifLuaTeX
\usepackage[bidi=basic]{babel}
\else
\usepackage[bidi=default]{babel}
\fi
\babelprovide[main,import]{english}
% get rid of language-specific shorthands (see #6817):
\let\LanguageShortHands\languageshorthands
\def\languageshorthands#1{}
% Manuscript styling
\usepackage{upgreek}
\captionsetup{font=singlespacing,justification=justified}

% Table formatting
\usepackage{longtable}
\usepackage{lscape}
% \usepackage[counterclockwise]{rotating}   % Landscape page setup for large tables
\usepackage{multirow}		% Table styling
\usepackage{tabularx}		% Control Column width
\usepackage[flushleft]{threeparttable}	% Allows for three part tables with a specified notes section
\usepackage{threeparttablex}            % Lets threeparttable work with longtable

% Create new environments so endfloat can handle them
% \newenvironment{ltable}
%   {\begin{landscape}\centering\begin{threeparttable}}
%   {\end{threeparttable}\end{landscape}}
\newenvironment{lltable}{\begin{landscape}\centering\begin{ThreePartTable}}{\end{ThreePartTable}\end{landscape}}

% Enables adjusting longtable caption width to table width
% Solution found at http://golatex.de/longtable-mit-caption-so-breit-wie-die-tabelle-t15767.html
\makeatletter
\newcommand\LastLTentrywidth{1em}
\newlength\longtablewidth
\setlength{\longtablewidth}{1in}
\newcommand{\getlongtablewidth}{\begingroup \ifcsname LT@\roman{LT@tables}\endcsname \global\longtablewidth=0pt \renewcommand{\LT@entry}[2]{\global\advance\longtablewidth by ##2\relax\gdef\LastLTentrywidth{##2}}\@nameuse{LT@\roman{LT@tables}} \fi \endgroup}

% \setlength{\parindent}{0.5in}
% \setlength{\parskip}{0pt plus 0pt minus 0pt}

% Overwrite redefinition of paragraph and subparagraph by the default LaTeX template
% See https://github.com/crsh/papaja/issues/292
\makeatletter
\renewcommand{\paragraph}{\@startsection{paragraph}{4}{\parindent}%
  {0\baselineskip \@plus 0.2ex \@minus 0.2ex}%
  {-1em}%
  {\normalfont\normalsize\bfseries\itshape\typesectitle}}

\renewcommand{\subparagraph}[1]{\@startsection{subparagraph}{5}{1em}%
  {0\baselineskip \@plus 0.2ex \@minus 0.2ex}%
  {-\z@\relax}%
  {\normalfont\normalsize\itshape\hspace{\parindent}{#1}\textit{\addperi}}{\relax}}
\makeatother

% \usepackage{etoolbox}
\makeatletter
\patchcmd{\HyOrg@maketitle}
  {\section{\normalfont\normalsize\abstractname}}
  {\section*{\normalfont\normalsize\abstractname}}
  {}{\typeout{Failed to patch abstract.}}
\patchcmd{\HyOrg@maketitle}
  {\section{\protect\normalfont{\@title}}}
  {\section*{\protect\normalfont{\@title}}}
  {}{\typeout{Failed to patch title.}}
\makeatother

\usepackage{xpatch}
\makeatletter
\xapptocmd\appendix
  {\xapptocmd\section
    {\addcontentsline{toc}{section}{\appendixname\ifoneappendix\else~\theappendix\fi\\: #1}}
    {}{\InnerPatchFailed}%
  }
{}{\PatchFailed}
\keywords{AI, Mental Health\newline\indent Word count: X}
\DeclareDelayedFloatFlavor{ThreePartTable}{table}
\DeclareDelayedFloatFlavor{lltable}{table}
\DeclareDelayedFloatFlavor*{longtable}{table}
\makeatletter
\renewcommand{\efloat@iwrite}[1]{\immediate\expandafter\protected@write\csname efloat@post#1\endcsname{}}
\makeatother
\usepackage{lineno}

\linenumbers
\usepackage{csquotes}
\makeatletter
\renewcommand{\paragraph}{\@startsection{paragraph}{4}{\parindent}%
  {0\baselineskip \@plus 0.2ex \@minus 0.2ex}%
  {-1em}%
  {\normalfont\normalsize\bfseries\typesectitle}}
\renewcommand{\subparagraph}[1]{\@startsection{subparagraph}{5}{1em}%
  {0\baselineskip \@plus 0.2ex \@minus 0.2ex}%
  {-\z@\relax}%
  {\normalfont\normalsize\bfseries\itshape\hspace{\parindent}{#1}\textit{\addperi}}{\relax}}
\makeatother

\ifLuaTeX
  \usepackage{selnolig}  % disable illegal ligatures
\fi

\title{Implementation Issues AI - Mental Healthcare}
\author{Anne-Kathrin Kleine\textsuperscript{1} \& \textsuperscript{}}
\date{}

\begin{document}
\maketitle

\hypertarget{relevant-terms}{%
\subsection{Relevant terms:}\label{relevant-terms}}

\begin{itemize}
\tightlist
\item
  The International Medical Device Regulators Forum (IMDRF) has defined \emph{Software} as a Medical Device as ``software intended to be used for medical purposes that performs its objectives without being part of a hardware medical device.''
\item
  \emph{Automation vs.~decision support tasks}:

  \begin{itemize}
  \tightlist
  \item
    Automation tasks are cases in which ``a machine operates independently to complete a task,'' whereas clinical decision support tasks are cases in which ``a machine is concerned with providing information or assistance to the primary agent responsible for task completion.''
  \end{itemize}
\end{itemize}

\hypertarget{issues-in-the-implementation-of-ai-in-mental-healthcare-practice}{%
\subsection{Issues in the implementation of AI in mental healthcare practice}\label{issues-in-the-implementation-of-ai-in-mental-healthcare-practice}}

\begin{itemize}
\tightlist
\item
  large amount of academic knowledge and developed algorithms not integrated into clinical care (Sendak et al., 2020)

  \begin{itemize}
  \tightlist
  \item
    ``This narrative review was unable to provide standard metrics of adoption, because many of the figures marketed by product developers have no peer-reviewed evidence'' (Sendak et al., 2020)
  \end{itemize}
\item
  Big data confidentiality (Aafjes-van Doorn et al., 2021)
\item
  Black box problems Kelly et al. (2019)
\item
  In addition, black-box predictive models combined with (similarly complex) explanatory methods may yield complicated decision pathways that increase the likelihood of human error (Chekroud et al., 2021)
\item
  ethical challenges:

  \begin{itemize}
  \tightlist
  \item
    responsibility (Chekroud et al., 2021)
  \item
    dehuminization (Chekroud et al., 2021)
  \item
    in clinical settings: transparency highly values - opposing black box problem (Chekroud et al., 2021)
  \item
    erronous outcomes for underrepresented groups (Chekroud et al., 2021)
  \item
    misuse of personal and sensitive data (Chekroud et al., 2021)
  \end{itemize}
\item
  diagnostic challenges (Lee et al., 2021)

  \begin{itemize}
  \tightlist
  \item
    Performance of supervised algorithms depends on the quality of the diagnostic labels used to train a model; Given the heterogeneity characteristic of mental illnesses, labels of disease states may not be specific enough to yield AI algorithms with high sensitivity and specificity

    \begin{itemize}
    \tightlist
    \item
      One possibility is to use ML algorithms to predict specific symptoms or functional consequences rather than diagnoses
    \item
      Another opportunity lies in leveraging the strength of deep neural networks that can operate without human oversight to identify novel biomarkers for detecting specific diseases (29)
    \item
      When the results of ML algorithms are published, they must include information regarding the quality of the data used to train the model as well as any potential biases in it, which is rarely done at present.
    \end{itemize}
  \end{itemize}
\item
  Specificity vs sensitivity tradeoffs:

  \begin{itemize}
  \tightlist
  \item
    e.g., However, the positive predictive value (PPV) (number of correctly predicted positive cases divided by the number of predicted positive cases) of prediction models for suicide attempts and deaths remains extremely low. In a systematic review of 17 studies, Belsher et al.~found a PPV of less than 1\% for suicide mortality despite good accuracy (greater than or equal to 80\%) {[}193{]}. In other words, ML algorithms still deliver a high rate of false alarms despite a high level of accuracy (Roth et al., 2021)

    \begin{itemize}
    \tightlist
    \item
      shown for multiple areas (assessment of suicide risk, depression, psychosis)
    \end{itemize}
  \item
    More complex ML models often have greater accuracy but lower interpretability
  \item
    Generally, there is a trade-off between explainablity and performance. For instance, a constrained linear or bilinear model will fit many of these criteria, but the linear model does not warrant a good performance. Additionally, a model that is potentially explainable does not guarantee explainability. For example, co-dependence of input variables may make explanations ambiguous (Chen et al., 2022)
  \end{itemize}
\item
  Furthermore, each mental disorder has various types of overlapping symptoms with varying degrees, bringing an additional challenge to uniquely define the disorder in psychiatry (unlike a clear cut in cardiology or oncology) (Chen et al., 2022)
\item
  many mental disorders have overlapping symptoms with other physical or mental disorders (Chen et al., 2022)
\end{itemize}

\hypertarget{issues-in-application-research}{%
\subsection{Issues in application research}\label{issues-in-application-research}}

\begin{itemize}
\tightlist
\item
  few studies test algorithms in independent samples Kelly et al. (2019)
\item
  when randomizing patients to algorithm-informed care or usual care, clinicians may override algorithm recommendations and choose alternative treatments (Chekroud et al., 2021)
\item
  Patients may refuse the algorithm-recommended treatment, or have restrictions to its use that were not contemplated by the decision support tool (e.g., prohibitive cost of therapy) (Chekroud et al., 2021)
\item
  In light of this, effect sizes for these interventions will often vary when applied in different settings (Chekroud et al., 2021)
\item
  the development of data-driven decision tools should be informed by extensive consultation and coproduction with the intended users, in order to implement models that maximize acceptability and compatibility with other clinical guidelines (i.e., risk management procedures, norms about safe dosage or titration of medications) (Chekroud et al., 2021)
\item
  fear of being substituted by AI systems?
\item
  research environments must encourage large-scale, collaborative, interdisciplinary consortia (Browning et al., 2020)
\item
  performance metrics:

  \begin{itemize}
  \tightlist
  \item
    The selected factors may include both specific computational properties such as parameter identifiability as well as practical features of an assay (e.g.~duration to complete, complexity) and clinical validity (e.g.~correlation with symptoms or treatment response) (Browning et al., 2020)
  \item
    Longitudinal observational studies may be used to assess whether an assay covaries with mental state changes or traits of interest and whether it has predictive validity, for example by predicting response to treatment (Browning et al., 2020)
  \item
    Regardless of whether the goal of using a computational assay is to predict a clinical outcome or to guide the development of a novel treatment, the efficacy of computationally informed approaches must ultimately be assessed in clinical trials. Such trials may, for example, randomly assign patients to be treated according to a predictive algorithm or standard treatment, or to receive a computationally informed intervention vs.~a control (Browning et al., 2020)
  \item
    difficulty of comparing different algorithms and AI systems (Kelly et al., 2019)
  \item
    ``products listed in Table 2 that predict the same outcome cannot be easily compared. Reporting of machine learning models often fails to follow establish best practices and model performance measures are not standardised across publications'' (Sendak et al., 2020)
  \item
    there is no current standard definition of accuracy and patient health outcomes against which to measure the products.
  \item
    metrics may not reflect clinical applicability: e.g., AUC not the most useful metric and difficult to understand by clinicians (Kelly et al., 2019)
  \item
    However, none of these measures ultimately reflect what is most important to patients, namely whether the use of the model results in a beneficial change in patient care Shah et al. (2019)
  \item
    possible solution: decision curve analysis
  \end{itemize}
\end{itemize}

\hypertarget{ways-out-and-forward}{%
\subsection{Ways out and forward}\label{ways-out-and-forward}}

\begin{itemize}
\tightlist
\item
  When conducted with care for ethical considerations, ML research can become an essential complement to traditional psychotherapy research (Chekroud et al., 2021)
\item
  highlight AI as a chance and addition to common practice (supporting, not substituting):

  \begin{itemize}
  \tightlist
  \item
    It is important to highlight that none of the identified ML applications were developed to replace the therapist, but instead were designed to advance the therapists' skills and treatment outcome (Chekroud et al., 2021)
  \end{itemize}
\item
  educating about limitations AND chances (Roth et al., 2021)
\end{itemize}

\hypertarget{multimodality-of-sources}{%
\subsubsection{Multimodality of sources}\label{multimodality-of-sources}}

\begin{itemize}
\tightlist
\item
  ML methods provide an opportunity for multimodal analyses of patient and therapist moment-by-moment changes in word use, speech, body movements, and physiological states, that are not (yet) usually considered in clinical decision making (Chekroud et al., 2021)
\item
  Illustrations include Instagram photographs to predict risk of developing depression (51), speech data to predict psychosis onset in high-risk youth (52), and identifying individuals with PTSD (53) (Lee et al., 2021)
\item
  Mental illnesses may be observable in online contexts, and social media data have been leveraged to predict diagnoses and relapses (51,72,76,77), with accuracies comparable to clinician assessments and screening surveys (78) (Lee et al., 2021)
\item
  leveraging ``big data'' from a longitudinal perspective offers a promising way to track the trajectories of neural phenotypes that have been rarely examined in previous cross-sectional studies of psychiatric disorders (Chen et al., 2022)
\item
  AI methodology can also incorporate both genetic and environmental risks (54), accounting for complex environment-gene interactions and psych-bio-social factors, particularly relevant in PTSD (55) (Lee et al., 2021)
\item
  Furthermore, AI methodologies are well-suited for deciphering patterns from longitudinal data (56), critical for honing the accuracy of diagnoses based on evolving psychiatric symptoms (Lee et al., 2021)
\item
  Lastly, AI methods may have a growing role in gathering sensitive and accurate data from patients. One study found that individuals were more forthcoming disclosing sensitive information with a computer system than with a person (57) (Lee et al., 2021)
\end{itemize}

\hypertarget{precision-psychiatry}{%
\subsubsection{Precision psychiatry}\label{precision-psychiatry}}

\begin{itemize}
\tightlist
\item
  finer grained diagnoses possible: First, AI approaches can bolster the ability to differentiate between diagnoses with similar initial clinical presentations but divergent treatment approaches (43) -- e.g., identifying bipolar versus unipolar depression based on brain imaging features (44), or differentiating between types of dementia using structural MRI scans (45) (Lee et al., 2021)
\item
  Secondly, data-driven AI methods can help identify novel disease subtypes based on heterogeneity of presentations, demographic features, and environmental factors (43). Examples include neurocognitive profiles in bipolar disorder (46), genetic profiles in schizophrenia (47), biomarker profiles in psychoses (48), and neuroimaging subtypes in depression (49) (Lee et al., 2021)
\item
  Thirdly, AI approaches can build models from unusual/novel data sources and reconcile data from multiple heterogeneous datastreams, e.g., EHR, behavioral data from digital phenotyping and wearable sensors, speech, social media feeds, neurophysiology, imaging, and genetics (50), to coalesce explanatory and mechanistic models of mental illness across self-report to molecular assessments (Lee et al., 2021)
\item
  that existing clinical diagnostic categories could misrepresent the causes underlying mental disturbance and the case-control study design has limited strength in delineating the significant clinical and neurobiological heterogeneity of psychiatric disorders (Chen et al., 2022)
\item
  The ultimate goal of RDoC is to find ``new ways of classifying psychiatric diseases based on multiple dimensions of biology and behavior'' (Chen et al., 2022)
\item
  Thanks to the advancement in cuttingedge techniques in ML/AI, psychiatrists and investigators now have an unprecedented opportunity to benefit from complex patterns in brain, behavior, and genes using machine learning tools (Chen et al., 2022)
\item
  Increasing evidence suggests that datadriven subtyping could drive novel neurobiological phenotypes associated with distinctive behavior and cognitive functioning
\item
  Chances for \emph{precision psychiatyry}: categorization of psychiatric patients into new data-driven subgroups (Roth et al., 2021)

  \begin{itemize}
  \tightlist
  \item
    less stigmatization
  \item
    homogenous disease classification, early diagnosis, prediction of disease trajectory, and tailored, more effective, safer, and predictable treatment, potentially at the individual level
  \end{itemize}
\item
  Clinical decision support (CDS) provides clinicians with knowledge (e.g., treatment guidelines) and patient-specific information (e.g., clinical and laboratory data), specifically selected and presented in a timely fashion, to enhance the quality of medical care (Roth et al., 2021)
\end{itemize}

\hypertarget{importance-of-practitioner-training}{%
\subsubsection{Importance of practitioner training:}\label{importance-of-practitioner-training}}

\begin{itemize}
\tightlist
\item
  To improve understanding, medical students and practising clinicians should be provided with an easily accessible AI curriculum to enable them to critically appraise, adopt and use AI tools safely in their practice (Kelly et al., 2019)
\item
  Thus, it will be important for psychotherapy researchers to become better-versed in the ML methods and how to interpret this research literature (Chekroud et al., 2021)
\item
  Accessible ML education and tool development is required to facilitate understanding and usage in the wider clinical research community. Besides formal education on ML in psychology graduate programs, it might also be helpful for psychotherapy researchers to attend (online and freely available) courses on ML (Chekroud et al., 2021)
\item
  Sendak et al.~(105) have proposed four phases of translation necessary to bridge this gap: design and development of ML products that can support clinical decision-making and are actionable; evaluation and validation; diffusion and scaling across settings such that the tools are more widely applicable; and continued monitoring and maintenance to remain current with clinical practice needs (Lee et al., 2021)
\item
  For instance, a classification function learned by the machine to predict a disease outcome would not only need to report a probability outcome but also need to address additional questions for the end-user: why is this outcome instead of the alternative? How reliable is the outcome? When does it fail if something is missing or misrepresented? When and why the prediction is wrong? Accordingly, a model with improved interpretability comes with parameter/structure/connectivity constraints and some prior domain knowledge (Chen et al., 2022)
\item
  The translation milestones (Sendak et al., 2020)

  \begin{itemize}
  \item
    To map between individual products and the translational path, milestones for each product are marked within four phases:
  \item
    \begin{enumerate}
    \def\labelenumi{\arabic{enumi})}
    \tightlist
    \item
      design and develop
    \end{enumerate}

    \begin{itemize}
    \tightlist
    \item
      The setting and funding of the team shapes many aspects of how the machine learning product is designed and developed.
    \item
      For example, in an academic setting it may be easier to cultivate collaborations across domains of expertise early on in the process. However, academic settings may have difficulty recruiting and retaining the technical talent required to productise complex technologies.
    \end{itemize}
  \item
    \begin{enumerate}
    \def\labelenumi{\arabic{enumi})}
    \setcounter{enumi}{1}
    \tightlist
    \item
      evaluate and validate
    \end{enumerate}

    \begin{itemize}
    \tightlist
    \item
      Clinical utility: can the product improve clinical care and patient outcomes?
    \item
      Statistical validity: Can the machine learning product perform well on metrics of accuracy, reliability, and calibration?
    \item
      Economic utility: Can there be a net benefit from the investment in the machine learning product?
    \end{itemize}
  \item
    \begin{enumerate}
    \def\labelenumi{\arabic{enumi})}
    \setcounter{enumi}{2}
    \tightlist
    \item
      diffuse and scale
    \end{enumerate}

    \begin{itemize}
    \tightlist
    \item
      diffuse and scale across settings, which requires special attention to deployment modalities, funding, and drivers of adoption.
    \item
      To scale, machine learning products must be able to ingest data from different EHR and must also support on-premise and cloud deployments. For this reason, many models are also distributed as stand-alone web applications that require manual entry to calculate risk.
    \end{itemize}
  \item
    \begin{enumerate}
    \def\labelenumi{\arabic{enumi})}
    \setcounter{enumi}{3}
    \tightlist
    \item
      continuing monitoring and maintenance
    \end{enumerate}

    \begin{itemize}
    \tightlist
    \item
      Data quality, population characteristics, and clinical practice change over time and impact the validity and utility of models.
    \item
      Model reliability and model updating are active fields of research and will be integral to ensure the robustness of machine learning products in clinical care.
    \end{itemize}
  \end{itemize}
\item
  machine learning technologies are referred to as products rather than models, recognising the significant effort required to productise and operationalise models that are often built primarily for academic purposes
\item
  The `inconvenient truth' of machine learning in healthcare was pointedly described as ``at present the algorithms that feature prominently in research literature are in fact not, for the most part, executable at the front lines of clinical practice.''
\item
  machine learning is initially expected to impact healthcare through augmenting rather than replacing clinical workflows
\item
  Machine learning technologies were included as case studies if they met two criteria: 1) they tackle a clinical problem using solely EHR data; and 2) they are evaluated and validated through direct integration with an EHR to demonstrate clinical, statistical, or economic utility
\item
  Case studies were selected amongst 1,672 presentations at 9 informatics and machine learning conferences between January 2018 and October 2019
\end{itemize}

\hypertarget{research-ideas}{%
\subsection{Research Ideas}\label{research-ideas}}

\hypertarget{focus-group-psychiatrists}{%
\subsubsection{Focus Group Psychiatrists}\label{focus-group-psychiatrists}}

\begin{itemize}
\tightlist
\item
  Educate about chances and limitations
\item
  Discuss implementation possibilities and difficulties
\end{itemize}

\hypertarget{meta-analysis-ai-performance-in-fields-for-which-none-exists-see-roth_etal21}{%
\subsubsection{Meta-analysis AI performance in fields for which none exists (see Roth et al. (2021))}\label{meta-analysis-ai-performance-in-fields-for-which-none-exists-see-roth_etal21}}

\begin{itemize}
\tightlist
\item
  PTSD
\item
  Delirium
\item
  Substance use
\end{itemize}

\hypertarget{ai-for-ehr-electronic-health-records-handling-possibilities-review}{%
\subsubsection{AI for EHR (electronic health records) handling: Possibilities (review)}\label{ai-for-ehr-electronic-health-records-handling-possibilities-review}}

\hypertarget{interviews-about-implementation-possibilities-and-issues-with-practitioners-health-care-specialists-data-scientists-patients}{%
\subsubsection{Interviews about implementation possibilities and issues with practitioners (health care specialists, data scientists, (patients?))}\label{interviews-about-implementation-possibilities-and-issues-with-practitioners-health-care-specialists-data-scientists-patients}}

\hypertarget{rcts-in-mental-health-applications}{%
\subsubsection{RCTs in mental health applications}\label{rcts-in-mental-health-applications}}

\begin{itemize}
\tightlist
\item
  Physical health: Randomized controlled trials (RCTs) and prospective studies can bridge this gap between theory and practice, more
  rigorously demonstrating that AI models can have a quantifiable, positive impact when deployed in real healthcare settings (Rajpurkar et al., 2022)
\item
  currently largely missing for mental health applications
\end{itemize}

\newpage

\hypertarget{references}{%
\section{References}\label{references}}

\hypertarget{refs}{}
\begin{CSLReferences}{1}{0}
\leavevmode\vadjust pre{\hypertarget{ref-aafjes-vandoorn_etal21}{}}%
Aafjes-van Doorn, K., Kamsteeg, C., Bate, J., \& Aafjes, M. (2021). A scoping review of machine learning in psychotherapy research. \emph{Psychotherapy Research}, \emph{31}(1), 92--116. \url{https://doi.org/10.1080/10503307.2020.1808729}

\leavevmode\vadjust pre{\hypertarget{ref-browning_etal20}{}}%
Browning, M., Carter, C. S., Chatham, C., Ouden, H. D., Gillan, C. M., Baker, J. T., Chekroud, A. M., Cools, R., Dayan, P., Gold, J., Goldstein, R. Z., Hartley, C. A., Kepecs, A., Lawson, R. P., Mourao-Miranda, J., Phillips, M. L., Pizzagalli, D. A., Powers, A., Rindskopf, D., \ldots{} Paulus, M. (2020). Realizing the {Clinical} {Potential} of {Computational} {Psychiatry}: {Report} {From} the {Banbury} {Center} {Meeting}, {February} 2019. \emph{Biological Psychiatry}, \emph{88}(2), e5--e10. \url{https://doi.org/10.1016/j.biopsych.2019.12.026}

\leavevmode\vadjust pre{\hypertarget{ref-chekroud_etal21}{}}%
Chekroud, A. M., Bondar, J., Delgadillo, J., Doherty, G., Wasil, A., Fokkema, M., Cohen, Z., Belgrave, D., DeRubeis, R., Iniesta, R., Dwyer, D., \& Choi, K. (2021). The promise of machine learning in predicting treatment outcomes in psychiatry. \emph{World Psychiatry}, \emph{20}(2), 154--170. \url{https://doi.org/10.1002/wps.20882}

\leavevmode\vadjust pre{\hypertarget{ref-chen_etal22}{}}%
Chen, Z. S., Prathamesh, Kulkarni, Galatzer-Levy, I. R., Bigio, B., Nasca, C., \& Zhang, Y. (2022). \emph{Modern {Views} of {Machine} {Learning} for {Precision} {Psychiatry}}. arXiv. \url{http://arxiv.org/abs/2204.01607}

\leavevmode\vadjust pre{\hypertarget{ref-kelly_etal19}{}}%
Kelly, C. J., Karthikesalingam, A., Suleyman, M., Corrado, G., \& King, D. (2019). Key challenges for delivering clinical impact with artificial intelligence. \emph{BMC Medicine}, \emph{17}(1), 195. \url{https://doi.org/10.1186/s12916-019-1426-2}

\leavevmode\vadjust pre{\hypertarget{ref-lee_etal21}{}}%
Lee, E. E., Torous, J., De Choudhury, M., Depp, C. A., Graham, S. A., Kim, H.-C., Paulus, M. P., Krystal, J. H., \& Jeste, D. V. (2021). Artificial {Intelligence} for {Mental} {Health} {Care}: {Clinical} {Applications}, {Barriers}, {Facilitators}, and {Artificial} {Wisdom}. \emph{Biological Psychiatry: Cognitive Neuroscience and Neuroimaging}, \emph{6}(9), 856--864. \url{https://doi.org/10.1016/j.bpsc.2021.02.001}

\leavevmode\vadjust pre{\hypertarget{ref-rajpurkar_etal22}{}}%
Rajpurkar, P., Chen, E., Banerjee, O., \& Topol, E. J. (2022). {AI} in health and medicine. \emph{Nature Medicine}, \emph{28}(1), 31--38. \url{https://doi.org/10.1038/s41591-021-01614-0}

\leavevmode\vadjust pre{\hypertarget{ref-roth_etal21}{}}%
Roth, C. B., Papassotiropoulos, A., Brühl, A. B., Lang, U. E., \& Huber, C. G. (2021). Psychiatry in the {Digital} {Age}: {A} {Blessing} or a {Curse}? \emph{International Journal of Environmental Research and Public Health}, \emph{18}(16), 8302. \url{https://doi.org/10.3390/ijerph18168302}

\leavevmode\vadjust pre{\hypertarget{ref-sendak_etal20}{}}%
Sendak, M. P., D'Arcy, J., Kashyap, S., Gao, M., Nichols, M., Corey, K., Ratliff, W., \& Balu, S. (2020). A {Path} for {Translation} of {Machine} {Learning} {Products} into {Healthcare} {Delivery}. \emph{EMJ Innovations}. \url{https://doi.org/10.33590/emjinnov/19-00172}

\leavevmode\vadjust pre{\hypertarget{ref-shah_etal19}{}}%
Shah, N. H., Milstein, A., \& Bagley, P., Steven C. (2019). Making {Machine} {Learning} {Models} {Clinically} {Useful}. \emph{JAMA}, \emph{322}(14), 1351--1352. \url{https://doi.org/10.1001/jama.2019.10306}

\end{CSLReferences}

% papaja Lua-filter additions

\shorttitle{AI Implementation in Mental Healthcare}

\authornote{

LMU

The authors made the following contributions. Anne-Kathrin Kleine: Conceptualization, Writing - Original Draft Preparation, Writing - Review \& Editing; : , .

Correspondence concerning this article should be addressed to Anne-Kathrin Kleine, . E-mail: \href{mailto:Anne-Kathrin.Kleine@psy.lmu.de}{\nolinkurl{Anne-Kathrin.Kleine@psy.lmu.de}}

}

\affiliation{\vspace{0.5cm}\textsuperscript{1} LMU\\\textsuperscript{} }

% End of papaja Lua-filter additions

\end{document}
