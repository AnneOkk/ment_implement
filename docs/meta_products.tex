% Options for packages loaded elsewhere
\PassOptionsToPackage{unicode}{hyperref}
\PassOptionsToPackage{hyphens}{url}
%
\documentclass[
  man]{apa7}
\usepackage{amsmath,amssymb}
\usepackage{lmodern}
\usepackage{iftex}
\ifPDFTeX
  \usepackage[T1]{fontenc}
  \usepackage[utf8]{inputenc}
  \usepackage{textcomp} % provide euro and other symbols
\else % if luatex or xetex
  \usepackage{unicode-math}
  \defaultfontfeatures{Scale=MatchLowercase}
  \defaultfontfeatures[\rmfamily]{Ligatures=TeX,Scale=1}
\fi
% Use upquote if available, for straight quotes in verbatim environments
\IfFileExists{upquote.sty}{\usepackage{upquote}}{}
\IfFileExists{microtype.sty}{% use microtype if available
  \usepackage[]{microtype}
  \UseMicrotypeSet[protrusion]{basicmath} % disable protrusion for tt fonts
}{}
\makeatletter
\@ifundefined{KOMAClassName}{% if non-KOMA class
  \IfFileExists{parskip.sty}{%
    \usepackage{parskip}
  }{% else
    \setlength{\parindent}{0pt}
    \setlength{\parskip}{6pt plus 2pt minus 1pt}}
}{% if KOMA class
  \KOMAoptions{parskip=half}}
\makeatother
\usepackage{xcolor}
\IfFileExists{xurl.sty}{\usepackage{xurl}}{} % add URL line breaks if available
\IfFileExists{bookmark.sty}{\usepackage{bookmark}}{\usepackage{hyperref}}
\hypersetup{
  pdftitle={Review AI - Mental Healthcare Products},
  pdfauthor={Anne-Kathrin Kleine1 \& },
  pdflang={en-EN},
  pdfkeywords={AI, Mental Health, Products},
  hidelinks,
  pdfcreator={LaTeX via pandoc}}
\urlstyle{same} % disable monospaced font for URLs
\usepackage{graphicx}
\makeatletter
\def\maxwidth{\ifdim\Gin@nat@width>\linewidth\linewidth\else\Gin@nat@width\fi}
\def\maxheight{\ifdim\Gin@nat@height>\textheight\textheight\else\Gin@nat@height\fi}
\makeatother
% Scale images if necessary, so that they will not overflow the page
% margins by default, and it is still possible to overwrite the defaults
% using explicit options in \includegraphics[width, height, ...]{}
\setkeys{Gin}{width=\maxwidth,height=\maxheight,keepaspectratio}
% Set default figure placement to htbp
\makeatletter
\def\fps@figure{htbp}
\makeatother
\setlength{\emergencystretch}{3em} % prevent overfull lines
\providecommand{\tightlist}{%
  \setlength{\itemsep}{0pt}\setlength{\parskip}{0pt}}
\setcounter{secnumdepth}{-\maxdimen} % remove section numbering
% Make \paragraph and \subparagraph free-standing
\ifx\paragraph\undefined\else
  \let\oldparagraph\paragraph
  \renewcommand{\paragraph}[1]{\oldparagraph{#1}\mbox{}}
\fi
\ifx\subparagraph\undefined\else
  \let\oldsubparagraph\subparagraph
  \renewcommand{\subparagraph}[1]{\oldsubparagraph{#1}\mbox{}}
\fi
\newlength{\cslhangindent}
\setlength{\cslhangindent}{1.5em}
\newlength{\csllabelwidth}
\setlength{\csllabelwidth}{3em}
\newlength{\cslentryspacingunit} % times entry-spacing
\setlength{\cslentryspacingunit}{\parskip}
\newenvironment{CSLReferences}[2] % #1 hanging-ident, #2 entry spacing
 {% don't indent paragraphs
  \setlength{\parindent}{0pt}
  % turn on hanging indent if param 1 is 1
  \ifodd #1
  \let\oldpar\par
  \def\par{\hangindent=\cslhangindent\oldpar}
  \fi
  % set entry spacing
  \setlength{\parskip}{#2\cslentryspacingunit}
 }%
 {}
\usepackage{calc}
\newcommand{\CSLBlock}[1]{#1\hfill\break}
\newcommand{\CSLLeftMargin}[1]{\parbox[t]{\csllabelwidth}{#1}}
\newcommand{\CSLRightInline}[1]{\parbox[t]{\linewidth - \csllabelwidth}{#1}\break}
\newcommand{\CSLIndent}[1]{\hspace{\cslhangindent}#1}
\ifLuaTeX
\usepackage[bidi=basic]{babel}
\else
\usepackage[bidi=default]{babel}
\fi
\babelprovide[main,import]{english}
% get rid of language-specific shorthands (see #6817):
\let\LanguageShortHands\languageshorthands
\def\languageshorthands#1{}
% Manuscript styling
\usepackage{upgreek}
\captionsetup{font=singlespacing,justification=justified}

% Table formatting
\usepackage{longtable}
\usepackage{lscape}
% \usepackage[counterclockwise]{rotating}   % Landscape page setup for large tables
\usepackage{multirow}		% Table styling
\usepackage{tabularx}		% Control Column width
\usepackage[flushleft]{threeparttable}	% Allows for three part tables with a specified notes section
\usepackage{threeparttablex}            % Lets threeparttable work with longtable

% Create new environments so endfloat can handle them
% \newenvironment{ltable}
%   {\begin{landscape}\centering\begin{threeparttable}}
%   {\end{threeparttable}\end{landscape}}
\newenvironment{lltable}{\begin{landscape}\centering\begin{ThreePartTable}}{\end{ThreePartTable}\end{landscape}}

% Enables adjusting longtable caption width to table width
% Solution found at http://golatex.de/longtable-mit-caption-so-breit-wie-die-tabelle-t15767.html
\makeatletter
\newcommand\LastLTentrywidth{1em}
\newlength\longtablewidth
\setlength{\longtablewidth}{1in}
\newcommand{\getlongtablewidth}{\begingroup \ifcsname LT@\roman{LT@tables}\endcsname \global\longtablewidth=0pt \renewcommand{\LT@entry}[2]{\global\advance\longtablewidth by ##2\relax\gdef\LastLTentrywidth{##2}}\@nameuse{LT@\roman{LT@tables}} \fi \endgroup}

% \setlength{\parindent}{0.5in}
% \setlength{\parskip}{0pt plus 0pt minus 0pt}

% Overwrite redefinition of paragraph and subparagraph by the default LaTeX template
% See https://github.com/crsh/papaja/issues/292
\makeatletter
\renewcommand{\paragraph}{\@startsection{paragraph}{4}{\parindent}%
  {0\baselineskip \@plus 0.2ex \@minus 0.2ex}%
  {-1em}%
  {\normalfont\normalsize\bfseries\itshape\typesectitle}}

\renewcommand{\subparagraph}[1]{\@startsection{subparagraph}{5}{1em}%
  {0\baselineskip \@plus 0.2ex \@minus 0.2ex}%
  {-\z@\relax}%
  {\normalfont\normalsize\itshape\hspace{\parindent}{#1}\textit{\addperi}}{\relax}}
\makeatother

% \usepackage{etoolbox}
\makeatletter
\patchcmd{\HyOrg@maketitle}
  {\section{\normalfont\normalsize\abstractname}}
  {\section*{\normalfont\normalsize\abstractname}}
  {}{\typeout{Failed to patch abstract.}}
\patchcmd{\HyOrg@maketitle}
  {\section{\protect\normalfont{\@title}}}
  {\section*{\protect\normalfont{\@title}}}
  {}{\typeout{Failed to patch title.}}
\makeatother

\usepackage{xpatch}
\makeatletter
\xapptocmd\appendix
  {\xapptocmd\section
    {\addcontentsline{toc}{section}{\appendixname\ifoneappendix\else~\theappendix\fi\\: #1}}
    {}{\InnerPatchFailed}%
  }
{}{\PatchFailed}
\keywords{AI, Mental Health, Products\newline\indent Word count: X}
\DeclareDelayedFloatFlavor{ThreePartTable}{table}
\DeclareDelayedFloatFlavor{lltable}{table}
\DeclareDelayedFloatFlavor*{longtable}{table}
\makeatletter
\renewcommand{\efloat@iwrite}[1]{\immediate\expandafter\protected@write\csname efloat@post#1\endcsname{}}
\makeatother
\usepackage{lineno}

\linenumbers
\usepackage{csquotes}
\makeatletter
\renewcommand{\paragraph}{\@startsection{paragraph}{4}{\parindent}%
  {0\baselineskip \@plus 0.2ex \@minus 0.2ex}%
  {-1em}%
  {\normalfont\normalsize\bfseries\typesectitle}}
\renewcommand{\subparagraph}[1]{\@startsection{subparagraph}{5}{1em}%
  {0\baselineskip \@plus 0.2ex \@minus 0.2ex}%
  {-\z@\relax}%
  {\normalfont\normalsize\bfseries\itshape\hspace{\parindent}{#1}\textit{\addperi}}{\relax}}
\makeatother

\ifLuaTeX
  \usepackage{selnolig}  % disable illegal ligatures
\fi

\title{Review AI - Mental Healthcare Products}
\author{Anne-Kathrin Kleine\textsuperscript{1} \& \textsuperscript{}}
\date{}


\shorttitle{Review AI Mental Health}

\authornote{

LMU

The authors made the following contributions. Anne-Kathrin Kleine: Conceptualization, Writing - Original Draft Preparation, Writing - Review \& Editing; : , .

Correspondence concerning this article should be addressed to Anne-Kathrin Kleine, . E-mail: \href{mailto:Anne-Kathrin.Kleine@psy.lmu.de}{\nolinkurl{Anne-Kathrin.Kleine@psy.lmu.de}}

}

\affiliation{\vspace{0.5cm}\textsuperscript{1} LMU\\\textsuperscript{} }

\begin{document}
\maketitle

\hypertarget{introduction}{%
\subsection{Introduction}\label{introduction}}

{[}INSERT: Explanation AI{]}

Mental healthcare has been slower to adopt AI technology than physical healthcare Jiang et al. (2017).
Still, the number of AI-powered mental health applications has been rising over the past years Nahavandi et al. (2022).
Similar to physical health applications, there exists a gap between the AI algorithms and tools developed and tested in research and the available products ready to be used by patients and healthcare practitioners (see Leeuwen et al., 2021).
Specifically, despite the indications of benefits associated with integrating AI into mental healthcare to enhance diagnosis, treatment, and clinical administration quality (Shatte et al., 2019), most of the tools and algorithms developed and tested in research have not (yet) made it into production Chekroud et al. (2021).
In fact, ``no FDA-approved or FDA-cleared AI applications currently exist in psychiatry'' (Lee et al., 2021, p. 5).
The lack of available products mainly concerns diagnostic and recommendation tools aiming at detecting psychological disorders and suggesting clinical treatment approaches (e.g., psychopharmacotherapy versus psychotherapy) (Chekroud et al., 2021).
In their scoping review of machine learning in psychotherapy research, Aafjes-van Doorn et al. (2021) identified 51 studies that developed and tested a machine learning algorithm aiming to classify or predict treatment process or outcome data or identify clusters in the patient or treatment data.
The authors conclude that current applications of machine learning in psychotherapy research provide a range of benefits for choosing appropriate treatment regimes, predicting treatment adherence, supporting therapist skill development, and predicting treatment response.
Shatte et al. (2019) identified 190 mental health tools aiming to detect and diagnose mental health conditions, 67 focused on prognosis, treatment and support, 26 on public health applications, and 17 on research and clinical administration.
These research findings do not align with the scope of marketed AI-based mental health products.
While most research focuses on detecting and diagnosing mental health conditions, the market is dominated by low-threshold wellness and treatment applications, such as chatbots and virtual agents, and sensor-data-based stress reduction applications {[}REF NEEDED{]}.
The main reasons for the lack of implementation into clinical practice include patient data confidentiality issues, explainability versus performance trade-offs, and the frequency of erroneous predictions (e.g., among underrepresented groups) Aafjes-van Doorn et al. (2021).
In addition, mental health interventions often rely on the relational bond formed with the patient and the direct observation of patient behaviors and emotions and are hesitant to rely on AI recommendations (Shatte et al., 2019). \textbackslash{}

Despite the public attention devoted to the risks and potential benefits of AI-based mental health applications {[}MAYBE SOME NEWS ARTICLES{]}, an overview of available AI-supported mental health products is currently lacking.
In addition, we lack insight into the gap between research findings on AI-based mental health tools and marketed products in the field.
Such an insight may provide a starting point for a systematic reduction of implementation barriers on the side of health care practitioners, application developers, and the general public, including patients suffering from mental health conditions.
The current systematic review proceeds in two steps to answer these questions.
First, we provide an overview of available AI-powered mental health products.
Herein, we include products developed for the general public, patients, and mental health practitioners.
We will search {[}BRIEFLY DESCRIBE SEARCH STRATEGY{]}\ldots{}
Second, we will compare the available products with the current state of AI-based research in detection and diagnosis, prognosis, treatment and support, public health applications, and clinical administration (Shatte et al., 2019). \textbackslash{}

\hypertarget{questions}{%
\subsection{Questions}\label{questions}}

\begin{itemize}
\tightlist
\item
  How to find products?
\item
  Narrow down focus? - depression and anxiety?
\end{itemize}

\newpage

\hypertarget{references}{%
\section{References}\label{references}}

\hypertarget{refs}{}
\begin{CSLReferences}{1}{0}
\leavevmode\vadjust pre{\hypertarget{ref-aafjes-vandoorn_etal21}{}}%
Aafjes-van Doorn, K., Kamsteeg, C., Bate, J., \& Aafjes, M. (2021). A scoping review of machine learning in psychotherapy research. \emph{Psychotherapy Research}, \emph{31}(1), 92--116. \url{https://doi.org/10.1080/10503307.2020.1808729}

\leavevmode\vadjust pre{\hypertarget{ref-chekroud_etal21}{}}%
Chekroud, A. M., Bondar, J., Delgadillo, J., Doherty, G., Wasil, A., Fokkema, M., Cohen, Z., Belgrave, D., DeRubeis, R., Iniesta, R., Dwyer, D., \& Choi, K. (2021). The promise of machine learning in predicting treatment outcomes in psychiatry. \emph{World Psychiatry}, \emph{20}(2), 154--170. \url{https://doi.org/10.1002/wps.20882}

\leavevmode\vadjust pre{\hypertarget{ref-chen_etal22}{}}%
Chen, Z. S., Prathamesh, Kulkarni, Galatzer-Levy, I. R., Bigio, B., Nasca, C., \& Zhang, Y. (2022). \emph{Modern {Views} of {Machine} {Learning} for {Precision} {Psychiatry}}. arXiv. \url{http://arxiv.org/abs/2204.01607}

\leavevmode\vadjust pre{\hypertarget{ref-jiang_etal17}{}}%
Jiang, F., Jiang, Y., Zhi, H., Dong, Y., Li, H., Ma, S., Wang, Y., Dong, Q., Shen, H., \& Wang, Y. (2017). Artificial intelligence in healthcare: Past, present and future. \emph{Stroke and Vascular Neurology}, \emph{2}(4), 230--243. \url{https://doi.org/10.1136/svn-2017-000101}

\leavevmode\vadjust pre{\hypertarget{ref-kelly_etal19}{}}%
Kelly, C. J., Karthikesalingam, A., Suleyman, M., Corrado, G., \& King, D. (2019). Key challenges for delivering clinical impact with artificial intelligence. \emph{BMC Medicine}, \emph{17}(1), 195. \url{https://doi.org/10.1186/s12916-019-1426-2}

\leavevmode\vadjust pre{\hypertarget{ref-lee_etal21}{}}%
Lee, E. E., Torous, J., De Choudhury, M., Depp, C. A., Graham, S. A., Kim, H.-C., Paulus, M. P., Krystal, J. H., \& Jeste, D. V. (2021). Artificial {Intelligence} for {Mental} {Health} {Care}: {Clinical} {Applications}, {Barriers}, {Facilitators}, and {Artificial} {Wisdom}. \emph{Biological Psychiatry: Cognitive Neuroscience and Neuroimaging}, \emph{6}(9), 856--864. \url{https://doi.org/10.1016/j.bpsc.2021.02.001}

\leavevmode\vadjust pre{\hypertarget{ref-vanleeuwen_etal21}{}}%
Leeuwen, K. G. van, Schalekamp, S., Rutten, M. J. C. M., Ginneken, B. van, \& Rooij, M. de. (2021). Artificial intelligence in radiology: 100 commercially available products and their scientific evidence. \emph{European Radiology}, \emph{31}(6), 3797--3804. \url{https://doi.org/10.1007/s00330-021-07892-z}

\leavevmode\vadjust pre{\hypertarget{ref-miller_brown18}{}}%
Miller, D. D., \& Brown, E. W. (2018). Artificial {Intelligence} in {Medical} {Practice}: {The} {Question} to the {Answer}? \emph{The American Journal of Medicine}, \emph{131}(2), 129--133. \url{https://doi.org/10.1016/j.amjmed.2017.10.035}

\leavevmode\vadjust pre{\hypertarget{ref-miller_polson19}{}}%
Miller, E., \& Polson, D. (2019). Apps, {Avatars}, and {Robots}: {The} {Future} of {Mental} {Healthcare}. \emph{Issues in Mental Health Nursing}, \emph{40}(3), 208--214. \url{https://doi.org/10.1080/01612840.2018.1524535}

\leavevmode\vadjust pre{\hypertarget{ref-nahavandi_etal22}{}}%
Nahavandi, D., Alizadehsani, R., Khosravi, A., \& Acharya, U. R. (2022). Application of artificial intelligence in wearable devices: {Opportunities} and challenges. \emph{Computer Methods and Programs in Biomedicine}, \emph{213}, 106541. \url{https://doi.org/10.1016/j.cmpb.2021.106541}

\leavevmode\vadjust pre{\hypertarget{ref-roth_etal21}{}}%
Roth, C. B., Papassotiropoulos, A., Brühl, A. B., Lang, U. E., \& Huber, C. G. (2021). Psychiatry in the {Digital} {Age}: {A} {Blessing} or a {Curse}? \emph{International Journal of Environmental Research and Public Health}, \emph{18}(16), 8302. \url{https://doi.org/10.3390/ijerph18168302}

\leavevmode\vadjust pre{\hypertarget{ref-sendak_etal20}{}}%
Sendak, M. P., D'Arcy, J., Kashyap, S., Gao, M., Nichols, M., Corey, K., Ratliff, W., \& Balu, S. (2020). A {Path} for {Translation} of {Machine} {Learning} {Products} into {Healthcare} {Delivery}. \emph{EMJ Innovations}. \url{https://doi.org/10.33590/emjinnov/19-00172}

\leavevmode\vadjust pre{\hypertarget{ref-shatte_etal19}{}}%
Shatte, A. B. R., Hutchinson, D. M., \& Teague, S. J. (2019). Machine learning in mental health: A scoping review of methods and applications. \emph{Psychological Medicine}, \emph{49}(9), 1426--1448. \url{https://doi.org/10.1017/S0033291719000151}

\leavevmode\vadjust pre{\hypertarget{ref-vaidyam_etal19}{}}%
Vaidyam, A. N., Wisniewski, H., Halamka, J. D., Kashavan, M. S., \& Torous, J. B. (2019). Chatbots and {Conversational} {Agents} in {Mental} {Health}: {A} {Review} of the {Psychiatric} {Landscape}. \emph{The Canadian Journal of Psychiatry}, \emph{64}(7), 456--464. \url{https://doi.org/10.1177/0706743719828977}

\end{CSLReferences}


\end{document}
