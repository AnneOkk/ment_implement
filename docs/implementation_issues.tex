% Options for packages loaded elsewhere
\PassOptionsToPackage{unicode}{hyperref}
\PassOptionsToPackage{hyphens}{url}
%
\documentclass[
  man]{apa7}
\usepackage{amsmath,amssymb}
\usepackage{lmodern}
\usepackage{iftex}
\ifPDFTeX
  \usepackage[T1]{fontenc}
  \usepackage[utf8]{inputenc}
  \usepackage{textcomp} % provide euro and other symbols
\else % if luatex or xetex
  \usepackage{unicode-math}
  \defaultfontfeatures{Scale=MatchLowercase}
  \defaultfontfeatures[\rmfamily]{Ligatures=TeX,Scale=1}
\fi
% Use upquote if available, for straight quotes in verbatim environments
\IfFileExists{upquote.sty}{\usepackage{upquote}}{}
\IfFileExists{microtype.sty}{% use microtype if available
  \usepackage[]{microtype}
  \UseMicrotypeSet[protrusion]{basicmath} % disable protrusion for tt fonts
}{}
\makeatletter
\@ifundefined{KOMAClassName}{% if non-KOMA class
  \IfFileExists{parskip.sty}{%
    \usepackage{parskip}
  }{% else
    \setlength{\parindent}{0pt}
    \setlength{\parskip}{6pt plus 2pt minus 1pt}}
}{% if KOMA class
  \KOMAoptions{parskip=half}}
\makeatother
\usepackage{xcolor}
\IfFileExists{xurl.sty}{\usepackage{xurl}}{} % add URL line breaks if available
\IfFileExists{bookmark.sty}{\usepackage{bookmark}}{\usepackage{hyperref}}
\hypersetup{
  pdftitle={Implementation Issues AI - Mental Healthcare},
  pdfauthor={Anne-Kathrin Kleine1 \& Anne-Kathrin Kleine1},
  pdflang={en-EN},
  pdfkeywords={AI, Mental Health},
  hidelinks,
  pdfcreator={LaTeX via pandoc}}
\urlstyle{same} % disable monospaced font for URLs
\usepackage{graphicx}
\makeatletter
\def\maxwidth{\ifdim\Gin@nat@width>\linewidth\linewidth\else\Gin@nat@width\fi}
\def\maxheight{\ifdim\Gin@nat@height>\textheight\textheight\else\Gin@nat@height\fi}
\makeatother
% Scale images if necessary, so that they will not overflow the page
% margins by default, and it is still possible to overwrite the defaults
% using explicit options in \includegraphics[width, height, ...]{}
\setkeys{Gin}{width=\maxwidth,height=\maxheight,keepaspectratio}
% Set default figure placement to htbp
\makeatletter
\def\fps@figure{htbp}
\makeatother
\setlength{\emergencystretch}{3em} % prevent overfull lines
\providecommand{\tightlist}{%
  \setlength{\itemsep}{0pt}\setlength{\parskip}{0pt}}
\setcounter{secnumdepth}{-\maxdimen} % remove section numbering
% Make \paragraph and \subparagraph free-standing
\ifx\paragraph\undefined\else
  \let\oldparagraph\paragraph
  \renewcommand{\paragraph}[1]{\oldparagraph{#1}\mbox{}}
\fi
\ifx\subparagraph\undefined\else
  \let\oldsubparagraph\subparagraph
  \renewcommand{\subparagraph}[1]{\oldsubparagraph{#1}\mbox{}}
\fi
\newlength{\cslhangindent}
\setlength{\cslhangindent}{1.5em}
\newlength{\csllabelwidth}
\setlength{\csllabelwidth}{3em}
\newlength{\cslentryspacingunit} % times entry-spacing
\setlength{\cslentryspacingunit}{\parskip}
\newenvironment{CSLReferences}[2] % #1 hanging-ident, #2 entry spacing
 {% don't indent paragraphs
  \setlength{\parindent}{0pt}
  % turn on hanging indent if param 1 is 1
  \ifodd #1
  \let\oldpar\par
  \def\par{\hangindent=\cslhangindent\oldpar}
  \fi
  % set entry spacing
  \setlength{\parskip}{#2\cslentryspacingunit}
 }%
 {}
\usepackage{calc}
\newcommand{\CSLBlock}[1]{#1\hfill\break}
\newcommand{\CSLLeftMargin}[1]{\parbox[t]{\csllabelwidth}{#1}}
\newcommand{\CSLRightInline}[1]{\parbox[t]{\linewidth - \csllabelwidth}{#1}\break}
\newcommand{\CSLIndent}[1]{\hspace{\cslhangindent}#1}
\ifLuaTeX
\usepackage[bidi=basic]{babel}
\else
\usepackage[bidi=default]{babel}
\fi
\babelprovide[main,import]{english}
% get rid of language-specific shorthands (see #6817):
\let\LanguageShortHands\languageshorthands
\def\languageshorthands#1{}
% Manuscript styling
\usepackage{upgreek}
\captionsetup{font=singlespacing,justification=justified}

% Table formatting
\usepackage{longtable}
\usepackage{lscape}
% \usepackage[counterclockwise]{rotating}   % Landscape page setup for large tables
\usepackage{multirow}		% Table styling
\usepackage{tabularx}		% Control Column width
\usepackage[flushleft]{threeparttable}	% Allows for three part tables with a specified notes section
\usepackage{threeparttablex}            % Lets threeparttable work with longtable

% Create new environments so endfloat can handle them
% \newenvironment{ltable}
%   {\begin{landscape}\centering\begin{threeparttable}}
%   {\end{threeparttable}\end{landscape}}
\newenvironment{lltable}{\begin{landscape}\centering\begin{ThreePartTable}}{\end{ThreePartTable}\end{landscape}}

% Enables adjusting longtable caption width to table width
% Solution found at http://golatex.de/longtable-mit-caption-so-breit-wie-die-tabelle-t15767.html
\makeatletter
\newcommand\LastLTentrywidth{1em}
\newlength\longtablewidth
\setlength{\longtablewidth}{1in}
\newcommand{\getlongtablewidth}{\begingroup \ifcsname LT@\roman{LT@tables}\endcsname \global\longtablewidth=0pt \renewcommand{\LT@entry}[2]{\global\advance\longtablewidth by ##2\relax\gdef\LastLTentrywidth{##2}}\@nameuse{LT@\roman{LT@tables}} \fi \endgroup}

% \setlength{\parindent}{0.5in}
% \setlength{\parskip}{0pt plus 0pt minus 0pt}

% Overwrite redefinition of paragraph and subparagraph by the default LaTeX template
% See https://github.com/crsh/papaja/issues/292
\makeatletter
\renewcommand{\paragraph}{\@startsection{paragraph}{4}{\parindent}%
  {0\baselineskip \@plus 0.2ex \@minus 0.2ex}%
  {-1em}%
  {\normalfont\normalsize\bfseries\itshape\typesectitle}}

\renewcommand{\subparagraph}[1]{\@startsection{subparagraph}{5}{1em}%
  {0\baselineskip \@plus 0.2ex \@minus 0.2ex}%
  {-\z@\relax}%
  {\normalfont\normalsize\itshape\hspace{\parindent}{#1}\textit{\addperi}}{\relax}}
\makeatother

% \usepackage{etoolbox}
\makeatletter
\patchcmd{\HyOrg@maketitle}
  {\section{\normalfont\normalsize\abstractname}}
  {\section*{\normalfont\normalsize\abstractname}}
  {}{\typeout{Failed to patch abstract.}}
\patchcmd{\HyOrg@maketitle}
  {\section{\protect\normalfont{\@title}}}
  {\section*{\protect\normalfont{\@title}}}
  {}{\typeout{Failed to patch title.}}
\makeatother

\usepackage{xpatch}
\makeatletter
\xapptocmd\appendix
  {\xapptocmd\section
    {\addcontentsline{toc}{section}{\appendixname\ifoneappendix\else~\theappendix\fi\\: #1}}
    {}{\InnerPatchFailed}%
  }
{}{\PatchFailed}
\keywords{AI, Mental Health\newline\indent Word count: X}
\DeclareDelayedFloatFlavor{ThreePartTable}{table}
\DeclareDelayedFloatFlavor{lltable}{table}
\DeclareDelayedFloatFlavor*{longtable}{table}
\makeatletter
\renewcommand{\efloat@iwrite}[1]{\immediate\expandafter\protected@write\csname efloat@post#1\endcsname{}}
\makeatother
\usepackage{lineno}

\linenumbers
\usepackage{csquotes}
\makeatletter
\renewcommand{\paragraph}{\@startsection{paragraph}{4}{\parindent}%
  {0\baselineskip \@plus 0.2ex \@minus 0.2ex}%
  {-1em}%
  {\normalfont\normalsize\bfseries\typesectitle}}
\renewcommand{\subparagraph}[1]{\@startsection{subparagraph}{5}{1em}%
  {0\baselineskip \@plus 0.2ex \@minus 0.2ex}%
  {-\z@\relax}%
  {\normalfont\normalsize\bfseries\itshape\hspace{\parindent}{#1}\textit{\addperi}}{\relax}}
\makeatother

\ifLuaTeX
  \usepackage{selnolig}  % disable illegal ligatures
\fi

\title{Implementation Issues AI - Mental Healthcare}
\author{Anne-Kathrin Kleine\textsuperscript{1} \& Anne-Kathrin Kleine\textsuperscript{1}}
\date{}


\shorttitle{AI Implementation in Mental Healthcare}

\authornote{

LMU

The authors made the following contributions. Anne-Kathrin Kleine: Conceptualization, Writing - Original Draft Preparation, Writing - Review \& Editing; Anne-Kathrin Kleine: Writing - Review \& Editing, Supervision.

Correspondence concerning this article should be addressed to Anne-Kathrin Kleine, . E-mail: \href{mailto:Anne-Kathrin.Kleine@psy.lmu.de}{\nolinkurl{Anne-Kathrin.Kleine@psy.lmu.de}}

}

\affiliation{\vspace{0.5cm}\textsuperscript{1} LMU\\\textsuperscript{} }

\begin{document}
\maketitle

\hypertarget{issues-in-the-implementation-of-ai-in-mental-healthcare-practice}{%
\subsection{Issues in the implementation of AI in mental healthcare practice}\label{issues-in-the-implementation-of-ai-in-mental-healthcare-practice}}

\begin{itemize}
\tightlist
\item
  Big data confidentiality (Aafjes-van Doorn et al., 2021)
\item
  Black box problems Chekroud et al. (2021)
\item
  In addition, black-box predictive models combined with (similarly complex) explanatory methods may yield complicated decision pathways that increase the likelihood of human error (Chekroud et al., 2021)
\item
  ethical challenges:

  \begin{itemize}
  \tightlist
  \item
    responsibility (Chekroud et al., 2021)
  \item
    dehuminization (Chekroud et al., 2021)
  \item
    in clinical settings: transparency highly values - opposing black box problem (Chekroud et al., 2021)
  \item
    erronous outcomes for underrepresented groups (Chekroud et al., 2021)
  \item
    misuse of personal and sensitive data (Chekroud et al., 2021)
  \end{itemize}
\end{itemize}

\hypertarget{issues-in-intervention-studies}{%
\subsection{Issues in intervention studies}\label{issues-in-intervention-studies}}

\begin{itemize}
\tightlist
\item
  few studies test algorithms in independent samples Chekroud et al. (2021)
\item
  when randomizing patients to algorithm-informed care or usual care, clinicians may override algorithm recommendations and choose alternative treatments (Chekroud et al., 2021)
\item
  Patients may refuse the algorithm-recommended treatment, or have restrictions to its use that were not contemplated by the decision support tool (e.g., prohibitive cost of therapy) (Chekroud et al., 2021)
\item
  In light of this, effect sizes for these interventions will often vary when applied in different settings (Chekroud et al., 2021)
\item
  the development of data-driven decision tools should be informed by extensive consultation and coproduction with the intended users, in order to implement models that maximize acceptability and compatibility with other clinical guidelines (i.e., risk management procedures, norms about safe dosage or titration of medications) (Chekroud et al., 2021)
\item
  fear of being substituted by AI systems?
\end{itemize}

\hypertarget{ways-out-and-forward}{%
\subsection{Ways out and forward}\label{ways-out-and-forward}}

\begin{itemize}
\tightlist
\item
  Thus, it will be important for psychotherapy researchers to become better-versed in the ML methods and how to interpret this research literature (Chekroud et al., 2021)
\item
  Accessible ML education and tool development is required to facilitate understanding and usage in the wider clinical research community. Besides formal education on ML in psychology graduate programs, it might also be helpful for psychotherapy researchers to attend (online and freely available) courses on ML (Chekroud et al., 2021)
\item
  When conducted with care for ethical considerations, ML research can become an essential complement to traditional psychotherapy research (Chekroud et al., 2021)
\item
  highlight AI as a chance and addition to commoon practice (supporting, not substituting):

  \begin{itemize}
  \tightlist
  \item
    It is important to highlight that none of the identified ML applications were developed to replace the therapist, but instead were designed to advance the therapists' skills and treatment outcome (Chekroud et al., 2021)
  \item
    ML methods provide an opportunity for multimodal analyses of patient and therapist moment-bymoment changes in word use, speech, body movements, and physiological states, that are not (yet) usually considered in clinical decision making (Chekroud et al., 2021)
  \end{itemize}
\end{itemize}

\newpage

\hypertarget{references}{%
\section{References}\label{references}}

\hypertarget{refs}{}
\begin{CSLReferences}{1}{0}
\leavevmode\vadjust pre{\hypertarget{ref-aafjes-vandoorn_etal21}{}}%
Aafjes-van Doorn, K., Kamsteeg, C., Bate, J., \& Aafjes, M. (2021). A scoping review of machine learning in psychotherapy research. \emph{Psychotherapy Research}, \emph{31}(1), 92--116. \url{https://doi.org/10.1080/10503307.2020.1808729}

\leavevmode\vadjust pre{\hypertarget{ref-chekroud_etal21}{}}%
Chekroud, A. M., Bondar, J., Delgadillo, J., Doherty, G., Wasil, A., Fokkema, M., Cohen, Z., Belgrave, D., DeRubeis, R., Iniesta, R., Dwyer, D., \& Choi, K. (2021). The promise of machine learning in predicting treatment outcomes in psychiatry. \emph{World Psychiatry}, \emph{20}(2), 154--170. \url{https://doi.org/10.1002/wps.20882}

\end{CSLReferences}


\end{document}
